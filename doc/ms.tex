\documentclass[a4paper, 12pt]{article}
\usepackage{amsmath}
\usepackage{amssymb}
\usepackage{booktabs} % For nicer tables
\usepackage{color}
\usepackage[left=2cm, right=2cm, bottom=3cm, top=2cm]{geometry}
\usepackage{graphicx}
\usepackage[utf8]{inputenc}
\usepackage{microtype}
\usepackage{natbib}
\usepackage{transparent}

\renewcommand{\arraystretch}{1.2} % More space between table rows

\title{Bayesian Inference from Time Series: Information Loss and the Whittle Likelihood}
\author{Brendon J. Brewer}
\date{2018}

\begin{document}
\maketitle

\abstract{\noindent Inference from a time series dataset can be performed
          either with the raw time series data, or after transformation to
          the fourier domain. In the latter case, Whittle's approximation to
          the likelihood function is computationally convenient. However, the
          question of how much information is lost remains open --- in many
          cases, the cost of extra computing time might be worth the benefits
          of preserving all of the information from the raw time series data.
          In this paper I investigate this question using numerical experiments
          to quantify the mutual information between a dataset and a parameter
          under the exact and the Whittle likelihoods.}

% Need this after the abstract
\setlength{\parindent}{0pt}
\setlength{\parskip}{1em}

\section{Introduction}

Just as probability theory (used in the `Bayesian' way)
quantifies the degree to which one statement or
proposition implies another \citep{knuth_skilling},
information theory describes the
degree to which resolving one question would resolve another
\citep{knuth_questions, vanerp}.

In a recent paper, I introduced a computational technique based on Nested
Sampling for estimating the entropy of probability distributions which are
usually only accessible through sampling, such as posterior distributions.

\subsection{Time series and the whittle likelihood}

Assume a quantity has been measured at $N$ equally-spaced time points
$\boldsymbol{t} = \{0, 1, 2, ..., N-1\}$, resulting in observations
$\boldsymbol{y} = \{y_0, y_1, ..., y_{N-1}\}$.
% Use GP framework

The whittle likelihood is
\begin{align}
p(\boldsymbol{y} | \boldsymbol{\theta})
    &\propto -\sum_{k=0}^{N/2 - 1}
         \left(S(\omega_k; \boldsymbol{\theta})
         + \frac{I(\omega_k; \boldsymbol{y})}
                {S(\omega_k; \boldsymbol{\theta})}\right)\label{eqn:whittle}
\end{align}
where $I(\omega_k; \boldsymbol{y})$ is the power spectrum of the data,
$S(\omega_k; \boldsymbol{\theta})$ is the power spectral density implied
by the model parameters, and the summation is over the angular frequencies
involved in the first half of the discrete fourier transform.
Equation~\ref{eqn:whittle} is equivalent to an exponential distribution
for the empirical power spectrum
$I(\omega_k; \boldsymbol{y})$ whose expectation is given by the
model power spectrum $S(\omega_k; \boldsymbol{\theta})$.

In astronomy, time series analysis of variable sources has often been performed
in the frequency domain \citep[e.g.][]{lomb, hatzes, helio}. However,
a number of researchers have applied existing time series models from
statistics \citep{brockwell_davis} or custom-made models to infer parameters
directly from the time series data, without any use of the frequency domain
\citep[e.g.][]{gregory_loredo, gregory, kelly, brewer_stello, farr}.

Methods that analyse the time series data directly are often less convenient than frequency-domain fits, and can have a much greater computational cost,
despite recent computational advances which have improved the matter
considerably \citep[e.g.][]{hodlr, celerite}. Yet the question remains ---
if we opt for the convenient frequency-domain methods, how much relevant
information are we throwing away? That is the focus of this paper.


\section{A Damped, Stochastically-Excited Oscillator}
Consider the following ordinary differential equation, which describes
a damped and excited oscillator:
\begin{align}
\frac{d^2y}{dt^2} + \frac{\omega_0}{Q}\frac{dy}{dt}
        + \omega_0^2\frac{dy}{dt} &= \epsilon(t).\label{eqn:ode}
\end{align}
The function $y(t)$ is the resulting signal, and the parameters $\omega_0$
are the angular frequency (that the signal would have if
the driving force $\epsilon(t)$ were zero) and $Q$ is the quality factor.
If the driving force $\epsilon(t)$ is only defined probabilistically (i.e.,
it is specified to be white noise) then Equation~\ref{eqn:ode} becomes a
stochastic differential equation whose solution is a probability distribution
for $y(t)$.

We assume that the probability distribution for $\epsilon(t)$ is indeed white
noise. In this case, the probability distribution for $y(t)$ given the
parameters is a gaussian process with mean function of zero and
covariance function given by
\begin{align}
C(\tau) &= A^2\left[\cos(\eta\omega_0\tau)
                + \frac{1}{2\eta Q}\sin(\eta\omega_0\tau)\right]
    \label{eqn:oscillation_covariance}
\end{align}
where $\eta = \sqrt{1 - 1/(4Q^2)}$, $\tau$ is a time separation,
and $Q > 1/2$ (i.e., we assume the oscillation is not overdamped).
A more general expression which allows for $Q \leq 1/2$
is given by \citet{anderson} and \citet{celerite}.
In the limit of large $Q$, the covariance function is approximately
the product of an oscillation and an exponential decay:
\begin{align}
C(\tau) &= A\exp\left(-\frac{\omega_0\tau}{2Q}\right)\cos(\omega_0\tau)
\end{align}
which was the form assumed by \citet{brewer_stello}.
The power spectral density corresponding to
Equation~\ref{eqn:oscillation_covariance} is
\begin{align}
S(\omega) &= \sqrt{\frac{2}{\pi}}
             \frac{S_0\omega_0^4}
                  {(\omega^2 - \omega_0^2)^2 + \omega_0^2\omega^2/Q^2}
\end{align}
where $S_0 = A/(\omega_0Q)$.

\subsection{Inferring the period}
The parameter $\omega_0$ is the angular frequency, related to the period
$T$ by
\begin{align}
T &= \frac{2\pi}{\omega_0}.
\end{align}
In this section we assume that it is the log-period $\log_{10}(T)$ that is
of interest, and that the observations consist of $N$ measurements
$\boldsymbol{y} = \{y_0, y_1, ..., y_{N-1}\}$,
at times $\boldsymbol{t} = \{0,1,2,...,N-1\}$, observed with noise:
\begin{align}
y_i &\sim \textnormal{Normal}(y(t_i), \sigma^2).
\end{align}
The noise level $\sigma$ was set to unity.

The unknown parameters are the oscillation parameters $(A, T, Q)$, to which
we assigned the following priors:
\begin{align}
\ln A &\sim \textnormal{Normal}(0, 1) \\
\log_{10} T &\sim \textnormal{Uniform}(\log_{10}(N) - 1, \log_{10}(N)) \\
\ln Q &\sim \textnormal{Uniform}(\ln 1, \ln 1000).
\end{align}
The prior for the amplitude was chosen so that it is likely to be of the same
order of magnitude as the noise level, and the prior for the period asserts that
there are between one and ten complete periods within the observation period.
I did not allow shorter periods than this, so the current paper does not
address the issue of aliasing (equivalently, multimodal posterior distributions
for the period).

We proceeded to compute the conditional entropy
$H(\log_{10}(\tau) | \boldsymbol{y})$ using the approach of \citet{brewer},
and hence the mutual information $I(\log_{10}(\tau) ; \boldsymbol{y})$.
The results are given in Table~\ref{tab:oscillation_results}.



\begin{table}[!ht]
\centering
\begin{tabular}{@{}l@{\hspace{3em}}l@{\hspace{2em}}l@{}}
\toprule
$N$         &       Likelihood      &  Mut. Inf. (nats) \\
\hline
20          &       Exact           & $1.930 \pm 0.055$ \\
50          &       Exact           & $2.391 \pm 0.056$ \\
100         &       Exact           & $2.605 \pm 0.053$ \\
200         &       Exact           &   \\
20          &       Whittle         & $1.491 \pm 0.055$ \\ 
50          &       Whittle         & $1.957 \pm 0.055$ \\
100         &       Whittle         & $2.052 \pm 0.055$ \\
200         &       Whittle         &   \\
\bottomrule
\end{tabular}
\caption{The mutual information between the data and the log-period, for
different numbers of datapoints $N$ and for different likelihoods (exact vs.
Whittle).
This is equal to the prior expected reduction in the entropy
in going from the prior to the posterior.\label{tab:oscillation_results}}
\end{table}


\begin{table}[!ht]
\centering
\begin{tabular}{@{}l@{\hspace{3em}}l@{}}
\toprule
$N$         &       $\Delta$ Mut. Inf. (nats) \\
\hline
20          &       $0.439 \pm 0.053$ \\
50          &       $0.434 \pm 0.056$ \\
100         &       $0.553 \pm 0.056$ \\
200         & \\
\bottomrule
\end{tabular}
\caption{The difference in mutual information between the exact and
Whittle likelihoods. This quantifies the size of the advantage of the exact
likelihood.
The error bars are smaller than what you might expect from
taking differences in Table~\ref{tab:oscillation_results}, due to
the common stream of datasets.\label{tab:oscillation_differences}}
\end{table}

\begin{thebibliography}{999}

\bibitem[Ambikasaran et al.(2016)]{hodlr}
Ambikasaran, Sivaram, Daniel Foreman-Mackey, Leslie Greengard, David W. Hogg, and Michael O’Neil. ``Fast direct methods for Gaussian processes.''
IEEE transactions on pattern analysis and machine intelligence 38, no. 2 (2016): 252-265.

\bibitem[Anderson et al.(1990)]{anderson}
Anderson, Edwin R., Thomas L. Duvall Jr, and Stuart M. Jefferies. ``Modeling of solar oscillation power spectra.'' The Astrophysical Journal 364 (1990): 699-705.

\bibitem[Brewer(2017)]{brewer}
Brewer, Brendon J. ``Computing Entropies with Nested Sampling.'' Entropy 19, no. 8 (2017): 422.

\bibitem[Brewer and Stello(2009)]{brewer_stello}
Brewer, Brendon J., and Dennis Stello. ``Gaussian process modelling of asteroseismic data.'' Monthly Notices of the Royal Astronomical Society 395, no. 4 
(2009): 2226-2233.

\bibitem[Brockwell and Davis(2013)]{brockwell_davis}
Brockwell, P.J. and Davis, R.A., 2013. Time series: theory and methods. Springer Science \& Business Media.

\bibitem[Christensen-Dalsgaard(2002)]{helio}
Christensen-Dalsgaard, Jørgen. ``Helioseismology.'' Reviews of Modern Physics 74, no. 4 (2002): 1073.

\bibitem[Farr et al.(2018)]{farr}
Farr, Will M., Benjamin JS Pope, Guy R. Davies, Thomas SH North, Timothy R. White, Jim W. Barrett, Andrea Miglio et al. ``Aldebaran b's temperate past uncovered in planet search data.'' arXiv preprint arXiv:1802.09812 (2018).

\bibitem[Foreman-Mackey et al.(2017)]{celerite}
Foreman-Mackey, Daniel, Eric Agol, Sivaram Ambikasaran, and Ruth Angus. ``Fast and scalable Gaussian process modeling with applications to astronomical time series.'' The Astronomical Journal 154, no. 6 (2017): 220.

\bibitem[Gregory(2005)]{gregory}
Gregory, Philip C.
``A Bayesian analysis of extrasolar planet data for HD 208487.''
In AIP Conference Proceedings, vol. 803, no. 1, pp. 139-145. AIP, 2005.

\bibitem[Gregory and Loredo(1992)]{gregory_loredo}
Gregory, P. C., and Thomas J. Loredo. ``A new method for the detection of a periodic signal of unknown shape and period.'' The Astrophysical Journal 398 (1992): 146-168.

\bibitem[Hatzes et al.(1993)]{hatzes}
Hatzes, A.P. and Cochran, W.D., 1993. Long-period radial velocity variations in three K giants. The Astrophysical Journal, 413, pp.339-348.

\bibitem[Kelly(2009)]{kelly}
Kelly, Brandon C., Jill Bechtold, and Aneta Siemiginowska. ``Are the variations in quasar optical flux driven by thermal fluctuations?.'' The Astrophysical Journal 698, no. 1 (2009): 895.

\bibitem[Knuth(2005)]{knuth_questions}
Knuth, K.H. Toward question-asking machines: The logic of questions and the inquiry calculus. In
Proceedings of the 10th International Workshop On Artificial Intelligence and Statistics, Barbados, 6–8
January 2005.

\bibitem[Knuth and Skilling(2012)]{knuth_skilling}
Knuth, Kevin H., and John Skilling. ``Foundations of inference.''
Axioms 1, no. 1 (2012): 38-73.

\bibitem[Lomb(1976)]{lomb}
Lomb, Nicholas R. ``Least-squares frequency analysis of unequally spaced data.'' Astrophysics and space science 39, no. 2 (1976): 447-462.

\bibitem[van Erp et al.(2017)]{vanerp}
van Erp, H. R., Ronald O. Linger, and Pieter HAJM van Gelder. ``Inquiry Calculus and the Issue of Negative Higher Order Informations.'' Entropy 19, no. 11 (2017): 622.

\end{thebibliography}

\end{document}

