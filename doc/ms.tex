\documentclass[a4paper, 12pt]{article}
\usepackage{amsmath}
\usepackage{amssymb}
\usepackage{booktabs} % For nicer tables
\usepackage{color}
\usepackage[left=2cm, right=2cm, bottom=3cm, top=2cm]{geometry}
\usepackage{graphicx}
\usepackage[utf8]{inputenc}
\usepackage{microtype}
\usepackage{natbib}
\usepackage{transparent}

\renewcommand{\arraystretch}{1.2} % More space between table rows

\title{Information Loss from the Whittle Likelihood}
\author{Brendon J. Brewer}
\date{2018}

\begin{document}
\maketitle

\abstract{\noindent I investigate the information content of a time series
          dataset for inferring a parameter of interest. }

% Need this after the abstract
\setlength{\parindent}{0pt}
\setlength{\parskip}{1em}

\section{A Damped, Stochastically-Excited Oscillator}
Consider the following ordinary differential equation, which describes
a damped and excited oscillator:
\begin{align}
\frac{d^2y}{dt^2} + \frac{\omega_0}{Q}\frac{dy}{dt}
        + \omega_0^2\frac{dy}{dt} &= \epsilon(t).\label{eqn:ode}
\end{align}
The function $y(t)$ is the resulting signal, and the parameters $\omega_0$
are the angular frequency (that the signal would have if
the driving force $\epsilon(t)$ were zero) and $Q$ is the quality factor.
If the driving force $\epsilon{t}$ is only defined probabilistically (i.e.,
it is specified to be white noise) then Equation~\ref{eqn:ode} becomes a
stochastic differential equation whose solution is a probability distribution
for $y(t)$.

\begin{table}[!ht]
\centering
\begin{tabular}{@{}l@{\hspace{3em}}l@{\hspace{2em}}l@{}}
\toprule
$N$         &       Likelihood      &  Mut. Inf. (nats) \\
\hline
50          &       Exact           & \\
50          &       Whittle         & $1.957 \pm 0.055$ \\
100         &       Exact           & $2.605 \pm 0.053$ \\
100         &       Whittle         & $2.052 \pm 0.055$ \\
200         &       Exact           &   \\
200         &       Whittle         &   \\
\bottomrule
\end{tabular}
\caption{\label{tab:oscillator_results}}
\end{table}


\begin{table}[!ht]
\centering
\begin{tabular}{@{}l@{\hspace{3em}}l@{}}
\toprule
$N$         &       $\Delta$ Mut. Inf. (nats) \\
\hline
50          & \\
100         &       $0.553 \pm 0.056$ \\
200         & \\
\bottomrule
\end{tabular}
\caption{The difference in mutual information between the exact and
Whittle likelihoods. This quantifies the size of the advantage of the exact
likelihood.
The error bars are smaller than what you might expect from
taking differences in Table~\ref{tab:oscillator_results}, due to
the common stream of datasets.\label{tab:oscillator_differences}}
\end{table}

\begin{thebibliography}{999}
\bibitem[Foreman-Mackey et al.(2017)]{celerite}
Foreman-Mackey, D., Agol, E., Ambikasaran, S. and Angus, R., 2017. Fast and scalable Gaussian process modeling with applications to astronomical time series. The Astronomical Journal, 154(6), p.220.
\end{thebibliography}

\end{document}

