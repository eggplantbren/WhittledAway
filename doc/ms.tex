\documentclass[a4paper, 12pt]{article}
\usepackage{amsmath}
\usepackage{amssymb}
\usepackage{color}
\usepackage[left=2cm, right=2cm, bottom=3cm, top=2cm]{geometry}
\usepackage{graphicx}
\usepackage[utf8]{inputenc}
\usepackage{microtype}
\usepackage{natbib}
\usepackage{transparent}

\title{Information Loss from the Whittle Likelihood}
\author{Brendon J. Brewer}
\date{2018}

\begin{document}
\maketitle

\abstract{\noindent I investigate the information content of a time series
          dataset for inferring a parameter of interest. }

% Need this after the abstract
\setlength{\parindent}{0pt}
\setlength{\parskip}{1em}

\section{A Damped, Stochastically-Excited Oscillator}

\begin{table}[!ht]
\centering
\begin{tabular}{|lll|}
\hline
$N$         &       {\em Likelihood}      &  {\em Mutual Info. (nats)} \\
\hline
100         &       Exact           & $2.605 \pm 0.053$ \\
100         &       Whittle         & $2.052 \pm 0.055$ \\
\hline
\end{tabular}
\caption{\label{tab:oscillator_results}}
\end{table}


\begin{table}[!ht]
\centering
\begin{tabular}{|ll|}
\hline
$N$         &       $\Delta$ {\em Mutual Info. (nats)} \\
\hline
100         &       $0.553 \pm 0.056$ \\
\hline
\end{tabular}
\caption{The error bars are smaller than what you might expect from
taking differences in Table~\ref{tab:oscillator_results}, due to
the common stream of datasets.\label{tab:oscillator_differences}}
\end{table}

\begin{thebibliography}{999}
\bibitem[Foreman-Mackey et al.(2017)]{celerite}
Foreman-Mackey, D., Agol, E., Ambikasaran, S. and Angus, R., 2017. Fast and scalable Gaussian process modeling with applications to astronomical time series. The Astronomical Journal, 154(6), p.220.
\end{thebibliography}

\end{document}

