\documentclass[a4paper, 12pt]{article}
\usepackage{amsmath}
\usepackage{amssymb}
\usepackage{booktabs} % For nicer tables
\usepackage{color}
\usepackage[left=2cm, right=2cm, bottom=3cm, top=2cm]{geometry}
\usepackage{graphicx}
\usepackage[utf8]{inputenc}
\usepackage{microtype}
\usepackage{natbib}
\usepackage{transparent}

\renewcommand{\arraystretch}{1.2} % More space between table rows

\title{Bayesian Inference from Time Series: Information Loss and the Whittle Likelihood}
\author{Brendon J. Brewer}
\date{2018}

\begin{document}
\maketitle

\abstract{\noindent I investigate the information content of a time series
          dataset for inferring a parameter of interest. }

% Need this after the abstract
\setlength{\parindent}{0pt}
\setlength{\parskip}{1em}

\section{Introduction}

In a recent paper, I introduced a computational technique based on Nested
Sampling for estimating the entropy of probability distributions which are
usually only accessible through sampling, such as posterior distributions.


\section{A Damped, Stochastically-Excited Oscillator}
Consider the following ordinary differential equation, which describes
a damped and excited oscillator:
\begin{align}
\frac{d^2y}{dt^2} + \frac{\omega_0}{Q}\frac{dy}{dt}
        + \omega_0^2\frac{dy}{dt} &= \epsilon(t).\label{eqn:ode}
\end{align}
The function $y(t)$ is the resulting signal, and the parameters $\omega_0$
are the angular frequency (that the signal would have if
the driving force $\epsilon(t)$ were zero) and $Q$ is the quality factor.
If the driving force $\epsilon(t)$ is only defined probabilistically (i.e.,
it is specified to be white noise) then Equation~\ref{eqn:ode} becomes a
stochastic differential equation whose solution is a probability distribution
for $y(t)$.

We assume that the probability distribution for $\epsilon(t)$ is indeed white
noise. In this case, the probability distribution for $y(t)$ given the
parameters is a gaussian process with mean function of zero and
covariance function given by
\begin{align}
C(\tau) &= A^2\left[\cos(\eta\omega_0\tau)
                + \frac{1}{2\eta Q}\sin(\eta\omega_0\tau)\right]
    \label{eqn:oscillation_covariance}
\end{align}
where $\eta = \sqrt{1 - 1/(4Q^2)}$, $\tau$ is a time separation,
and $Q > 1/2$ (i.e., we assume the oscillation is not overdamped).
A more general expression which allows for $Q \leq 1/2$
is given by \citet{anderson} and \citet{celerite}.
In the limit of large $Q$, the covariance function is approximately
the product of an oscillation and an exponential decay:
\begin{align}
C(\tau) &= A\exp\left(-\frac{\omega_0\tau}{2Q}\right)\cos(\omega_0\tau)
\end{align}
which was the form assumed by \citet{brewer_stello}.
The power spectral density corresponding to
Equation~\ref{eqn:oscillation_covariance} is
\begin{align}
S(\omega) &= \sqrt{\frac{2}{\pi}}
             \frac{S_0\omega_0^4}
                  {(\omega^2 - \omega_0^2)^2 + \omega_0^2\omega^2/Q^2}
\end{align}
where $S_0 = A/(\omega_0Q)$.

\subsection{Inferring the period}
The parameter $\omega_0$ is the angular frequency, related to the period
$T$ by
\begin{align}
T &= \frac{2\pi}{\omega_0}.
\end{align}
In this section we assume that it is the log-period $\log_{10}(T)$ that is
of interest, and that the observations consist of $N$ measurements
$\boldsymbol{y} = \{y_0, y_1, ..., y_{N-1}\}$,
at times $\boldsymbol{t} = \{0,1,2,...,N-1\}$, observed with noise:
\begin{align}
y_i &\sim \textnormal{Normal}(y(t_i), \sigma^2).
\end{align}
The noise level $\sigma$ was set to unity.

The unknown parameters are the oscillation parameters $(A, T, Q)$, to which
we assigned the following priors:
\begin{align}
\ln A &\sim \textnormal{Normal}(0, 1) \\
\log_{10} T &\sim \textnormal{Uniform}(\log_{10}(N) - 1, \log_{10}(N)) \\
\ln Q &\sim \textnormal{Uniform}(\ln 1, \ln 1000).
\end{align}
We proceeded to compute the conditional entropy
$H(\log_{10}(\tau) | \boldsymbol{y})$ using the approach of \citet{brewer},
and hence the mutual information $I(\log_{10}(\tau) ; \boldsymbol{y})$.
The results are given in Table~\ref{tab:oscillation_results}.



\begin{table}[!ht]
\centering
\begin{tabular}{@{}l@{\hspace{3em}}l@{\hspace{2em}}l@{}}
\toprule
$N$         &       Likelihood      &  Mut. Inf. (nats) \\
\hline
20          &       Exact           & \\
20          &       Whittle         & $1.491 \pm 0.055$\\ 
50          &       Exact           & \\
50          &       Whittle         & $1.957 \pm 0.055$ \\
100         &       Exact           & $2.605 \pm 0.053$ \\
100         &       Whittle         & $2.052 \pm 0.055$ \\
200         &       Exact           &   \\
200         &       Whittle         &   \\
\bottomrule
\end{tabular}
\caption{The mutual information between the data and the log-period, for
different numbers of datapoints $N$ and for different likelihoods (exact vs.
Whittle).
This is equal to the prior expected reduction in the entropy
in going from the prior to the posterior.\label{tab:oscillator_results}}
\end{table}


\begin{table}[!ht]
\centering
\begin{tabular}{@{}l@{\hspace{3em}}l@{}}
\toprule
$N$         &       $\Delta$ Mut. Inf. (nats) \\
\hline
50          & \\
100         &       $0.553 \pm 0.056$ \\
200         & \\
\bottomrule
\end{tabular}
\caption{The difference in mutual information between the exact and
Whittle likelihoods. This quantifies the size of the advantage of the exact
likelihood.
The error bars are smaller than what you might expect from
taking differences in Table~\ref{tab:oscillator_results}, due to
the common stream of datasets.\label{tab:oscillator_differences}}
\end{table}

\begin{thebibliography}{999}
\bibitem[Anderson et al.(1990)]{anderson}
Anderson, Edwin R., Thomas L. Duvall Jr, and Stuart M. Jefferies. ``Modeling of solar oscillation power spectra.'' The Astrophysical Journal 364 (1990): 699-705.

\bibitem[Brewer(2017)]{brewer}
Brewer, Brendon J. ``Computing Entropies with Nested Sampling.'' Entropy 19, no. 8 (2017): 422.

\bibitem[Brewer and Stello(2009)]{brewer_stello}
Brewer, Brendon J., and Dennis Stello. ``Gaussian process modelling of asteroseismic data.'' Monthly Notices of the Royal Astronomical Society 395, no. 4 (2009): 2226-2233.

\bibitem[Foreman-Mackey et al.(2017)]{celerite}
Foreman-Mackey, Daniel, Eric Agol, Sivaram Ambikasaran, and Ruth Angus. ``Fast and scalable Gaussian process modeling with applications to astronomical time series.'' The Astronomical Journal 154, no. 6 (2017): 220.
\end{thebibliography}

\end{document}

